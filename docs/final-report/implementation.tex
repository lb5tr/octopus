\section{Implementacja}
Implementacja projektu, ze względu na rodzaj przedswięwzięcia został podzielona na dwie niezależne części.

Część kliencka, wysyłająca zapytania, odbierająca odpowiedzi na zapytania oraz odbierająca stan gry. Część kliencka odpowiedziala jest również za rysowanie aktualnego stanu gry oraz interfejsu użytkownika.

Część serwerowa wykonująca wszsytkie obliczenia, odbierająca zapytania, przetwarzająca je i rozyłąjąca je do użytkowników.

W rozdziale tym przedstawiono również specyfikację protokołu wykorzystywanego do komunikacji między stacją klienta a serwerem.

\subsection{Model obiektowy}
W ninejszym podrozdziale przedstawiono schemat obiektów zaimplementowany w aplikacji. Rysunek \ref{fig:main-obj} przedstawia diagram głównej część obkietów modelujących system, na rysunku \ref{fig:misc-obj} natomiast, pokazano obiekty, nie mające bezpośredniej relacji z główną częścią części serwerowej.
\begin{figure}[ht]
    \centering
    \includegraphics[width=0.8\textwidth]{imgs/main-obj.png}
    \caption{Diagram obiektów głównej części systemu.}
    \label{fig:main-obj}
\end{figure}

\begin{figure}[ht]
    \centering
    \includegraphics[width=0.8\textwidth]{imgs/misc-obj.png}
    \caption{Diagram pozostałych obiektów}
    \label{fig:misc-obj}
\end{figure}

Ze względu na charakter działania części klienckiej, która nie przechowuje żadnego stanu, nie został zaimplementowanny żaden konkretny model obiektowy.

\subsection{Protokół komunikacyjny}

Korzystając z aplikacji tworzonych z wykorzystaniem języka JavaScript, niesposób wykorzystać pełnie jego możliwości bez znajomości JSON. JSON jest notacją reprezentacji obiektów w języku, nietylko JavaScript, ale wszsytkich tych, które dostarczają możwliość obsługi tego standardu, który opiera się na prostym, hierarchicznym formacie tekstowym.

Wsparcie dla JSON w języku JavaScript jest zapewninone dzięki bibliotece standardowej, w języku Common Lisp należy wykorzystać pakiet \emph{CL-JSON}. Rysunek \ref{fig:jsonexample} przedstawia przykładowy obiekt w notacji JSON.
\begin{figure}[ht]
    \centering
    \begin{verbatim}
      {
        name : "Jan",
        lastName : "Kowalski",
        emails : [jan@kowalski.pl, janek@games.com],
        pet :
        {
          name : "reksio",
          species "C. Lupus",
          age: 4
        }
     }
      \end{verbatim}
    \caption{Przykładowa reprezentacja obiektu w formacie JSON}
    \label{fig:jsonexample}
\end{figure}
W przypadku prostych protokołów, z łatwością można wykorzystać JSON w zastępswie za XML.

W przypadku zaimplementowanego projektu, komunikaty dzielą się na dwa typy:
\begin{itemize}
  \item Wiadomości klienta (\emph{client message}).
  \item Wiadomości serwera (\emph{server message}).
\end{itemize}

\subsubsection{Wiadomość klienta}
Ogólna budowa rządania klienta została przedstawiona na rysunku \ref{fig:jsonclient}.
\begin{figure}[ht]
    \centering
    \begin{verbatim}
      {
        userId: <<ID klienta lub NULL>>,
        messageType: <<Rodzaj rządania>>,
        payload: <<obiekt wymagany przez typ wiadomości>
      }
      \end{verbatim}
    \caption{Reprezentacja obiektu zapytania klienta w notacji JSON.}
    \label{fig:jsonclient}
\end{figure}
Dozwolone rządania, wraz z opisem i wymaganym obiektem zostały przedstawione w tabeli \ref{tab:clientmessagestypes}

\begin{table}[h]
  \centering
  \begin{tabular}{ |p{1.5cm}|p{6cm}|p{6cm}| }
    \hline
    \textbf{Typ wiadomości} & \textbf{Opis} & \textbf{Rodzaj obiektu} \\ \hline
    login & Próba zalogowania użytkownika & Obiekt przesyłający nazwe użytkownika oraz skrót SHA1 jego hasła. \\
    create & Utwórz kanał gry & Obiekt podobny do obiektu \emph{channel}. Z wymaganymi polami \emph{name} i {capacity} oraz opcjonalnym polem \emph{password}. \\
    register & Zarejestrowanie użytkownika & Tożsame z wiadomością typu login \\
    join & Dołącz do rozgrywki & Obiekt zawierający nazwę porządanego kanału \\
    event & Wydarzenie jakie zostało wygenerowane przez użytkownika & Typ wydarzenia, np. nacisnięcie lewej strzałki, spacji itp. \\
    list & Wylistuj wszystkie dostępne kanały na serwerze & ignorowane \\
    logout & Wylogowanie użytkownika & ignorowane \\
    \hline
  \end{tabular}
  \caption{Przydział prac poszczególnym członkom zespołu}
  \label{tab:clientmessagestypes}
\end{table}
\subsubsection{Wiadomość serwera}
Wiadomości serwera można podzielić na dwie grupy:
\begin{itemize}
  \item Odpowiedzi na rządania.
  \item Cyklicznie rozyłany stan rozgrywki (do wszystkich użytkowników na danym kanale).
\end{itemize}

Odpowiedź na rządanie ma format tożsamy z formatem wiadomości klienta z tym, że pole \emph{messageType} może przyjąć jedną z dwóch wartości \emph{ok} lub \emph{error} a pole \emph{uid} jest nieobecne. W przypadku \emph{ok}, pole \emph{payload} niesie wartość wyniku rządania klienta. Jeżeli pole \emph{messageType} ma wartość \emph{error} serwer, pole \emph{payload} zawiera obiekt przedstawiony na rysunku \ref{fig:errors}
\begin{figure}[ht]
    \centering
    \begin{verbatim}
      {
        errorCode : <<Numer>>,
        errorDescription : <<Opis błędu>>
      }
      \end{verbatim}
    \caption{Reprezentacja obiektu zapytania klienta w notacji JSON.}
    \label{fig:errors}
\end{figure}

Stan rozysłany do użytkowników zawiera następujące pola, opisane w tabeli \ref{tab:state-fields}.

\begin{table}[ht]
  \centering
  \begin{tabular}{ |p{3cm}|p{8cm}| }
    \hline
    \textbf{Nazwa pola} & \textbf{Opis} \\ \hline
    \emph{players} & Lista użytkowników na kanale, łącznie z ich pozycją, rotacją i przynależnością do drużyny \\
    \emph{scoreYellow} & Punky drużyny żółtej \\
    \emph{scoreBlue} & Punky drużyny niebieskiej \\
    \emph{ballInstance} & Pozycja piłki \\
    \hline
  \end{tabular}
  \caption{Przydział prac poszczególnym członkom zespołu}
  \label{tab:state-fields}
\end{table}

\subsection{Cześć serwerowa}
Implementacja serwera, za sprawą wykorzystania języka Common Lisp została zaprogramowana korzystając z dwóch paradygmatów programowania. Funkcyjnym, za sprawą natury języka Lisp i obiektowym, za sprawą CLOS\footnote{\url{https://en.wikipedia.org/wiki/Common_Lisp_Object_System}}.

Przepływ danych w funkcjach jest następujący:
\begin{enumerate}
\item Dane od klienta docierają do funkcji obsługującej dane przychodzące od klientów. Funkcja jest implementacją funkcji \emph{resource-received-text} interfejsu \emph{ws-resource}.
\item Ponieważ wiadomość jest zserializowana do postaci tekstowej, zformatowaj do notacji JSON\footnote{\url{http://www.json.org/}}, należy ją zdeserializować korzystając z funkcji \emph{json-to-client-message}. W zależności od wartości pola \emph{messageType} wiadomość zostanie zserialowana do dane obiektu CLOS. Informacje o tym, jaki typ obiektu wybrać znajdują się w liście asocjacyjnej \emph{*message-payload-alist*}.
\item Obiekt zwócony przez poprzednią funkcję trafia do funkcji \emph{dispatch-message}, która również w zależności od \emph{messageType} wywołuje odpowiednią funkcję obsługującą zapytanie.
\item Funkcja obsługująca zapytanie zwraca tzn. listę własności (\emph{property list}) definiującą wynik obsługi zdażenia. Lista nasępnie trafia do funkcji \emph{response-with} która serializuje listę do obiektu \emph{server-message} definiującą rodzaj odpowiedzi, wynik lub ewentualny kod błędu w przypadku niepowodzenia.
\item Obiekt jest następnie serializowana do JSON i wysyłana do klienta korzystając z funkcji \emph{write-to-client-text}.
\end{enumerate}
Rysunek \ref{fig:serverflow} wizualizuje proces przetwarzania zapytań przez serwer.

Ponieważ serwer obsługuje szereg kanałów, jak i wielu użytkowników oraz oba te obiekty są zmienne w ilości, należy przechowywać je w kontenerach. Podstawowym kontenertem jest lista, jednak oczywiście złożoność dodawania, wyszukiwania oraz innych operacji ma zbyt wysoką złożoność. Aby rozwiązać ten problem wykorzystano strukturę tablicy haszującej dostępnej w bibliotece standardowej języka Common Lisp.
\begin{figure}[ht]
    \centering
    \includegraphics[width=0.5\textwidth]{imgs/serverflow.png}
    \caption{Przepływ danych po stronie serwerowej aplikacji.}
    \label{fig:serverflow}
\end{figure}

\subsection{Cześć kliencka}
Strukura interfejsu użytkownika, zdefiniowana przy użyciu języka HTML, reaguje na akcje użytkownika kożystając z języka JavaScript. Funkcje obsługujące zdarzenia połączone są z przyciskami kożystając z biblioteki jQuery\footnote{\url{http://jquery.com/}}. Na uwagę zasługuje funkcja \emph{dispatch}, która jest wysyłana za każdym razem gdy dane są zwracane od serwera. Funkcja, w zależności od tego jaka funkcja zostałą wywołana przez interfejsa użytkownika, wywoła funkcję która obsługuje odpowiedź na zapytanie. Jeżeli serwer odpowie wiadomością sugerującą nie powdodzenie w realizacji rządania wysłąnego przez klienta, funkcja obsługująca odpowiedź nie zostanie wywołana, a jedynie zostanie przedstawiona wiadomość z opisem błędu na interfejsie użytownika.

Rysunek \ref{fig:clientflow} prezentuje, jak wygląda proces przetwarzania akcji klienta.

\begin{figure}[ht]
    \centering
    \includegraphics[width=0.6\textwidth]{imgs/clientflow.png}
    \caption{Poszczególne aktywności generowane podczas przetwarzania danych klienta}
    \label{fig:clientflow}
\end{figure}

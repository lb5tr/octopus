\documentclass{beamer}
\usetheme{Warsaw}
\usepackage{polski}
\usepackage[utf8]{inputenc}
\usepackage{verbatim}
\usepackage{graphicx}
\title
    {Octopus}
    \subtitle{multiplayer, casual, haxball clone}
    \subsubtitle{Prezentacja nr 3}
    \author[]
           {Bujok Mikołaj \and Kubiak Jakub \\ \and Szymon Miękus\and Nikodem Adrian}
           \date[\today]
                {Projekt zespołowy 2013}
                \subject{Informatyka}

                \begin{document}
                %first frame
                \frame{\titlepage}
                \begin{frame}
                  \frametitle{Agenda}
                  \begin{itemize}
                  \item Po iteracji
                  \item Demo
                  \item Nadchodząca iteracja
                  \item Q\&A
                  \end{itemize}
                \end{frame}

                % frame
                \begin{frame}
                  \frametitle{Po iteracji}
                  \begin{itemize}
                  \item Wszystkie zaplanowane funkcjonalności zostały zaimplementowane
                    \begin{itemize}
                    \item Przetwarzanie wydarzeń od użytkowników
                    \item Implantacja silnika graficznego oraz fizycznego
                    \item Implementacja interfejsu użytkownika
                    \end{itemize}
                  \item Zebrana wiedza o projekcie
                    \begin{itemize}
                    \item Aktualnie wszystkie funkcjonalności są implementowane na czas
                    \item Estymacja czasowa funkcjonalności związanych z częścią serwerową jest w 50\% zaniżona (nieznacznie)
                    \end{itemize}
                  \end{itemize}
                \end{frame}

                % frame
                \begin{frame}
                  \frametitle{Demo}

                \end{frame}
                % frame
                \begin{frame}

                  \frametitle{Nadchodząca iteracja}
                  \begin{itemize}
                    \item Końcowa implementacja rozgrywki
                    \item Opcja rejestracji użytkownika
                    \item Dopracowanie interfejsu użytkownika
                    \item Rozszerzenie testów o testy UI
                  \end{itemize}
                \end{frame}
                \begin{frame}
                  \frametitle{Q\&A}
                \end{frame}

                \end{document}
